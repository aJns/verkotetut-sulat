\section{Suunnitelu}
Tarkoituksena oli tehdä sulautettu laite joka
\begin{enumerate}
\item voi ohjata LEDiä
\item voidaan liittää verkkoon
\item mahdollistaa LEDin ohjauksen verkkosivun kautta tai fyysistä nappia
painamalla.
\end{enumerate}
Vaatimuksena oli myös että piirilevy on itsetehty ja että siinä käytetään
mikrokontrolleria joka yhdistetään nettiin jonkunlaisen modeemin kautta
käyttäen AT-komentoja. Valitettavasti en itse ollut tietoinen tästä viimeisestä
vaatimuksesta, vaan ohjelmoin ESP8266 piirin suoraan. Tämän seurauksena
selostuksessa on puoli sivua ylimääräistä kohdassa~\ref{sec:extra}. Huomaamatta
jäi myös vaatimus fyysisestä painonapista, mutta sen lisääminen ei
onneksi ollut kovin iso työ.

Heti kun kuulin harjoitustyöstä halusin kokeilla käyttää ESP8266 moduulia
itsenäisesti ilman ohjaavaa mikrokontrolleria. Halusin myös ohjelmoida piirin
suoraan, enkä käyttää välissä käyttöjärjestelmää tai kolmannen osapuolen
firmwarea, kuten esimerkiksi NodeMcu:ta.\cite{nodemcu}

\section{Piirilevy ja komponentit}

\section{Ohjelmisto}

\section{ESP8266 moduulin käyttäminen ilman mikrokontrolleria}
\label{sec:extra}

\section{Yhteenveto}
